%Símbolos comuns
\DeclareMathOperator{\Mi}{\,\,\geqslant\,\,}
\DeclareMathOperator{\mi}{\,\,\leqslant\,\,}
\DeclareMathOperator{\mif}{\preccurlyeq}
\DeclareMathOperator{\Mif}{\succcurlyeq}
\DeclareMathOperator{\subnvazio}{\underset{^{\neq \varnothing}}{\subset}}
\usepackage{tipa}
\DeclareMathOperator{\vargamma}{\Large\text{\textramshorns}}

%Trigonométricas
\DeclareMathOperator{\sen}{\operatorname{sen}}

% Math symbol font matha
\DeclareFontFamily{U}{matha}{\hyphenchar\font45}
\DeclareFontShape{U}{matha}{m}{n}{
	<5> <6> <7> <8> <9> <10> gen * matha
	<10.95> matha10 <12> <14.4> <17.28> <20.74> <24.88> matha12
}{}
\DeclareSymbolFont{matha}{U}{matha}{m}{n}
\DeclareFontSubstitution{U}{matha}{m}{n}
% Math symbol font mathb
\DeclareFontFamily{U}{mathx}{\hyphenchar\font45}
\DeclareFontShape{U}{mathx}{m}{n}{
	<5> <6> <7> <8> <9> <10>
	<10.95> <12> <14.4> <17.28> <20.74> <24.88>
	mathx10
}{}
\DeclareSymbolFont{mathx}{U}{mathx}{m}{n}
\DeclareFontSubstitution{U}{mathx}{m}{n}

% Symbol definition
\DeclareMathDelimiter{\vvvert}{0}{matha}{"7E}{mathx}{"17}
\newcommand{\trinorma}[1]{\left\vvvert {#1}\right\vvvert}

% Cálculo
\DeclareMathOperator{\del}{\chapterial}
\newcommand{\classe}[1]{\mathscr{C}^{#1}}
\newcommand{\simples}[2]{\displaystyle \dfrac{{\td}{#1}}{{\td}{#2}}}
\newcommand{\parcial}[2]{\displaystyle\dfrac{\partial{#1}}{\partial{#2}}}
\newcommand{\pardois}[3]{\displaystyle\dfrac{\partial^2{#1}}{\partial{#2}\partial{#3}}}
\newcommand{\Lim}[2]{\displaystyle\lim_{#1\to#2}}
\newcommand{\Int}[2]{\displaystyle\int\limits_{#1}^{#2} }
\newcommand{\de}[1]{{\operatorname{d}}{#1}}

% Complexos
\newcommand{\con}[1]{\overline{#1}}
\newcommand{\cis}[1]{\text{cis}\left({#1}\right)}
\newcommand{\ordem}[2]{{\operatorname{ord}}\left({#1},{#2}\right)}
\newcommand{\res}[2]{{\operatorname{Res}}\left({#1},{#2}\right)}

%Abreviações

\newcommand{\evt}{\textsc{evt}}

\newcommand{\vazio}{\varnothing}

\newcommand{\somadir}[3]{
	\underset{{#1}}{\overset{{#2}}{\displaystyle\bigoplus}}\, {#3}
}

\newcommand{\inv}[1]{{#1}^{-1}}

\newcommand{\e}{\hspace{10px}\text{ e } \hspace{10px}}
\newcommand{\tenpx}{\hspace{10px}}

\newcommand{\sigalg}{$\sigma$-álgebra}

% Letras Gregas bbold

\newcommand{\GGamma}{\text{\reflectbox{\rotatebox[origin=c]{180}{$\mathbb L$}}}}
\newcommand{\LLambda}{\text{\reflectbox{\rotatebox[origin=c]{180}{\reflectbox{$\mathbb V$}}}}}
\newcommand{\Eexists}{\text{\reflectbox{$\mathbb E$}}}

\newcommand{\DDelta}{
	\mathrlap{\Delta \hspace{0.105cm}}{
		\begin{tikzpicture}
			\draw (0,0) -- (0,0.000001);
			\draw  (0.0582,0.01 - 0.004) -- 
			(0.0582+0.33*0.31779334981768276,
			0.33*0.6237045669318574- 0.004);
		\end{tikzpicture}\hspace{0.105cm}
	}
}

% Funtores
\usepackage{rotating}
\usepackage{scalerel}

% \covfunctor{F}{A}{B}{X}{Y}{Z}{W}{f}{g}{x}{y} =
% F : A -----→ B
%     X |----→ Y x
%     |        | -
%    f|       g| |
%     ↓        ↓ ↓
%     Z |----→ W y

%This is a first attemp with only 7 arguments
\newcommand{\funtordia}[7]{%
	\begin{tikzcd}[row sep={0.22em}, column sep={0.01em}, ampersand replacement=\&]
		#1: \& {#2} \arrow[rr] \&[1cm]\& {#3} \&   \\
		\& #5 \arrow[d,"#4"'] \arrow[rr, mapsto] \& \& #1(#5) \arrow[d,"#1(#4)"'] \& #7 \arrow[d,mapsto]\\[5mm]
		\& #6 \arrow[rr,mapsto]                     \& \& #1(#6)         \& #7\cdot #4
	\end{tikzcd}
}

%This is taken from the egreg's answer in the link you posted
\ExplSyntaxOn
\NewDocumentCommand{\NewWeirdCommand}{mm}
{% #1 = command to define, #2 = replacement text
	\cs_new:Npn #1 ##1
	{
		\tl_set:Nn \l__covfun_args_tl { ##1 }
		#2
	}
}
\NewDocumentCommand{\Arg}{m}
{
	\tl_item:Nn \l__covfun_args_tl { #1 }
}

\tl_new:N \l__covfun_parse_args_tl
\ExplSyntaxOff


%--------------------------
\NewWeirdCommand{\covfunctor}{%
	\begin{tikzcd}[row sep={0.22em}, column sep={0.01em}, ampersand replacement=\&]
		\Arg{1} \colon \& {\Arg{2}} \arrow[rr] \&[1cm]\& {\Arg{3}} \&   \\
		\& \Arg{4} \arrow[d,"\Arg{8}"'] \arrow[rr, mapsto] \& \& \Arg{5} \arrow[d,"\Arg{9}"'] \& \Arg{10} \arrow[d,mapsto]\\[5mm]
		\& \Arg{6} \arrow[rr,mapsto]                     \& \& \Arg{7}         \& \Arg{11}
	\end{tikzcd}
}
\NewWeirdCommand{\contfunctor}{%
	\begin{tikzcd}[row sep={0.22em}, column sep={0.01em}, ampersand replacement=\&]
		\Arg{1} \colon \& {\Arg{2}} \arrow[rr] \&[1cm]\& {\Arg{3}} \&   \\
		\& \Arg{4} \arrow[d,"\Arg{8}"'] \arrow[rr, mapsto] \& \& \Arg{5} \& \Arg{11} \\[5mm]
		\& \Arg{6} \arrow[rr,mapsto]                     \& \& \Arg{7} \arrow[u,"\Arg{9}"']  \& \Arg{10} \arrow[u,mapsto]
	\end{tikzcd}
}

% Função
\NewWeirdCommand{\function}{%
	\begin{tikzcd}[row sep={0.22em}, column sep={0.01em}, ampersand replacement=\&]
		{\Arg{1}}: \& {\Arg{2}} \arrow[rr] \& [0.5cm] \& {\Arg{3}} \& {\hphantom{\Arg{1}}} \\
		\& {\Arg{4}} \arrow[rr, mapsto] \&  \& {\Arg{5}} \&
	\end{tikzcd}
}


\newcommand{\consep}[1]{{#1}^{\textsc{sep}}}